\documentclass{article}


\usepackage[utf8]{inputenc}
\usepackage[english]{babel}
\usepackage{biblatex}
\addbibresource{biblio.bib}
\bibliography{biblio}


\title{Visualisation of big knowledge graphs}
\author{Flora Helmers, Mahsa Niazi}
\begin{document}
\maketitle

\section*{Introduction}
Posing the problem:
huge graphs 
visualisation (limits of 2D)


\section{Structure of the code}
rdf graph
-> 
structure of graphs
-> 
layout = (obj file )
-> 
obj file to be read in polyscope
-> 
adding elements to be read 

\section{Short presentation of the algorithm}
Presented in \cite{gajer00}. 

\section{What data structure to keep on the implementation}
\subsection{RDF  graphs}

\subsection{Intermediary structures}
several elements need to be stored every time 
- the filtration (store set of elements)
- the neighbors
- the positions 
- the local temperature `heat` 
//questions where to store them ? directly accessible or in a file to be written

First step : store them in vector<int>. 
    for the filtration

Second step: optimize with malloc

Third step : polyscope structure 


\subsection{Final result}


\printbibliography[
heading=bibintoc,
title={Bibliography}
]
\end{document}